\documentclass[11pt,a4paper]{article}
\usepackage[utf8]{inputenc}
\usepackage[margin=0.75in]{geometry}
\usepackage[hidelinks]{hyperref}
\usepackage{enumitem}
% \usepackage{fontawesome}
\usepackage{fontawesome5}
\usepackage{titlesec}
\usepackage{academicons}

% % --- Font: Source Sans Pro via Type1 (works with pdfLaTeX)
% \usepackage[default,scale=1.0]{sourcesanspro} % body + sans by default
% \renewcommand{\familydefault}{\sfdefault}      % make sans the default family
% \usepackage[scaled]{helvet}
\usepackage[default,scale=0.95]{opensans}

% (optional) math that pairs reasonably with sans body
\usepackage{newtxsf} % sans math; or use newtxtext/newtxmath for serif math


\usepackage[
    backend=biber,
    style=ieee,      % or 'ieee' / 'apa' / 'numeric-comp' depending on preference
    sorting=ydnt,          % Year (descending), name, title
    maxbibnames=99,        % show all authors instead of "et al."
    maxcitenames=2,        % optional: limit names in in-text citations
    giveninits=false,      % show full first names (set true for initials)
    uniquename=false,      % don't force initials to distinguish same names
    doi=false,             % suppress DOI (optional)
    url=false              % suppress URL unless needed
]{biblatex}
\addbibresource{references.bib}

\titleformat{\section}{\Large\bfseries}{\thesection}{1em}{}[\titlerule]
\titlespacing{\section}{0pt}{12pt}{6pt}

\begin{document}

\begin{center}
{\LARGE\textbf{Pavel Popov}}\\[0.5em]
\href{https://github.com/paavalipopov}{\faGithub\ github.com/paavalipopov} \quad
\href{https://www.linkedin.com/in/pavel-popov-83b69a123/}{\faLinkedin\ Pavel Popov} \quad
\href{https://scholar.google.com/citations?user=4alUD5UAAAAJ}{\aiGoogleScholar\ Pavel Popov}
\href{mailto:paavali.popov@gmail.com}{\faEnvelope\  paavali.popov@gmail.com}
\end{center}

\section*{Skills}
\begin{itemize}
[leftmargin=*,noitemsep]
\item \textbf{Programming \& Tools:} Python, PyTorch, Scikit-learn, Pandas, MATLAB, Git, Docker, SLURM, SQL, mongoDB.
\item \textbf{Machine Learning:} Deep learning architecture design, time series analysis, causal inference, explainable AI, DDP.
\item \textbf{Data mining:} Frequent pattern analysis, feature engineering, data visualization.
\item \textbf{Domain Expertise:} Neuroimaging, solid state physics, signal processing.

\end{itemize}

\section*{Education}
\textbf{PhD in Computer Science} \hfill 2022--05/2027\\
Georgia State University \hfill
Specialization in Machine Learning \& Neuroimaging Analysis\\

\textbf{MS in Applied Mathematics and Physics} \hfill 2018--2020\\
Moscow Institute of Physics and Technology \hfill Specialization in Spintronics\\

\textbf{BS in Applied Mathematics and Physics} \hfill 2014--2018\\
Moscow Institute of Physics and Technology \hfill Specialization in Solid State Physics

\section*{Experience}
\textbf{TReNDS Center : Graduate Research Assistant} \hfill 2022--Now
\begin{itemize}[leftmargin=*,noitemsep]
    \item Developed a robust DL classifier for medical diagnostics of functional MRI scans with SOTA performance and 10x speedup to existing benchmarks.
    \item Designed an interpretable DL simulator of brain activation dynamics for mining brain connectivity patterns and markers of brain disorders.
    \item Developed tools for training and deploying DL models for neuroimaging analysis using distributed systems.
    \item Collaborated with clinicians and radiologists to implement ML tools in clinical practice.
    \item Research project management, technical presentations.
\end{itemize}

\textbf{Institute of Radioengineering and Electronics : Research Assistant} \hfill 2017--2020
\begin{itemize}[leftmargin=*,noitemsep]
    \item Studied magnetic waves and complex geometry waveguides for potential use in novel logic device architectures.
    \item Theoretically described a voltage mechanism for control of THz magnetization dynamics in antiferromagnetic spin-torque oscillators.
    \item Conducted numerical simulations of magnetization dynamics in magnetic crystals.
\end{itemize}

\section*{Open Source Projects}
\textbf{\href{https://pypi.org/project/ml4fmri/}{ml4fmri}: Python toolkit for benchmarking DL models for fMRI analysis}
\begin{itemize}[leftmargin=*,noitemsep]
    \item Unified 13 DL models for multivatiate time series analysis under a standardized training API.
    \item Automated an all-model benchmarking pipeline for any input time series for reproducible research.
\end{itemize}

\textbf{\href{https://pypi.org/project/brainbow/}{brainbow} — Neuroimaging visualization utility}
\begin{itemize}[leftmargin=*,noitemsep]
    \item Created a command-line tool for visualizing obscure neuroimaging data formats.
\end{itemize}

\newpage
\section*{Publications}
\nocite{meanMLP}
\nocite{PhysRevAppl2020}
\nocite{JMMM2019}
\nocite{RiEl2018}
\nocite{Safin2020}
\nocite{Safin2020a}
\nocite{zafar}
\nocite{decifra_mlsp}


\printbibliography[
heading=none, 
sorting=ydnt
]
    
\end{document}
